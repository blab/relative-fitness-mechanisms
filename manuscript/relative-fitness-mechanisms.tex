%Preamble
\documentclass[11pt,oneside,letterpaper]{article}

% graphicx package, useful for including eps and pdf graphics
\usepackage{graphicx}
\DeclareGraphicsExtensions{.pdf,.png,.jpg}

% basic packages
\usepackage{color}
\usepackage{parskip}
\usepackage{float}

% make links a more pleasing color and don't box them
\definecolor{blue}{rgb}{0.85,0.70,0.99}
\usepackage[hidelinks]{hyperref}
\hypersetup{colorlinks=true,linkcolor=black,citecolor=black,urlcolor=blue}

% text layout
\usepackage{geometry}
\geometry{textwidth=15cm} % 15.25cm for single-space, 16.25cm for double-space
\geometry{textheight=22cm} % 22cm for single-space, 22.5cm for double-space

% helps to keep figures from being orphaned on a page by themselves
\renewcommand{\topfraction}{0.85}
\renewcommand{\textfraction}{0.1}

% bold the 'Figure #' in the caption and separate it with a period
% Captions will be left justified
\usepackage[labelfont=bf,labelsep=period,font=small]{caption}

% review layout with double-spacing
%\usepackage{setspace}
%\doublespacing
%\captionsetup{labelfont=bf,labelsep=period,font=doublespacing}

% cite package, to clean up citations in the main text
\usepackage{cite}

% Remove brackets from numbering in list of References
\renewcommand\refname{\large References}
\makeatletter
\renewcommand{\@biblabel}[1]{\quad#1.}
\makeatother

\usepackage{authblk}
\renewcommand\Authands{ \& }
\renewcommand\Authfont{\normalsize \bf}
\renewcommand\Affilfont{\small \normalfont}
\makeatletter
\renewcommand\AB@affilsepx{, \protect\Affilfont}
\makeatother

% notation
\usepackage{amsmath}
\usepackage{amssymb}

\renewcommand{\vec}[1]{\boldsymbol{#1}}
\newcommand{\Prob}{\mathbb{P}}
\newcommand{\wt}{\text{wt}}
\newcommand{\var}{\text{var}}
\newcommand{\vac}{\text{vac}}


\title{Transmission mechanisms determine relative fitness and frequency dynamics of viral variants}

\author[1,2,*]{Marlin D.\ Figgins}
\author[1,3]{Trevor Bedford}
\affil[1]{Vaccine and Infectious Disease Division, Fred Hutchinson Cancer Research Center, Seattle, WA, USA}
\affil[2]{Department of Applied Mathematics, University of Washington, Seattle, WA, USA}
\affil[3]{Howard Hughes Medical Institute, Seattle, WA, USA}
\affil[*]{Corresponding author: mfiggins@uw.edu}
\date{\today}
%Figure out author affilations...
% https://maehler.se/blog/2013/11/01/authors-and-affiliations-in-latex

\begin{document}

\maketitle

\begin{abstract}
    Viral sequencing efforts have ...
    Classifying variants into coarse fitness groups have allowed models of
    Models for frequency dynamics are becoming important as they are useful for describing the rise and fall of viral variants at course levels.
    However, these models are typically divorced from a direct interpretation at the level of transmission.
    We generalize the idea of frequency dynamic models to exponentially growing populations and show that relative fitness of variants can be interpreted using compartment models of infectious diseases.
    We then use these models to highlight several complications of frequency dynamic analysis for infectious disease variants and attribution of particular fitness mechanisms from frequency data alone. 
\end{abstract}

\section*{Introduction}

COVID-19 pandemic $\rightarrow$ Desire to quantify the rate at which novel variants are rising in frequency.

It has been shown that variant transmission advantages may differ geographically and temporally though simple models assume this transmission advantage is constant in time and space.
Further, existing models which do allow for variation in variant transmission advantages often do not have a mechanistic underpinning for why these transmission advantages exist and may persist.
These non-mechanistic transmission advantages are useful for real-time situation analysis though more work must be done to develop theory for how various mechanisms contribute to population change.

We introduce a theory of relative fitness and transmission advantages for exponentially growing populations.
We then apply this to several ordinary differential equation models of epidemics to assess how different transmission mechanisms may contribute to variant turnover.

(Point from here is just to analyze how these different mechanisms correspond to our existing models for transmission advantages and how this affects our ability to forecast).

\section*{Results}

\subsection*{Exponentially growing populations to frequency dynamics}%

We consider a population consisting of $V$ exponentially-growing variants
\begin{align*}
    \frac{d I_{v}}{d t} = r_{v}(t) I_{v}(t), \quad v = 1,2, \ldots, V.
\end{align*}

This differential equation has known solution

\begin{align*}
I_{v}(t) = I_{v}(0) \exp\left( \int_{0}^{t} r_{v}(s) ds\right),
\end{align*}
where $I_{v}(0)$ is the initial number of infectious individuals of variant $v$. 
Writing the frequency of variant $v$ in the population as  $f_{v}(t) = I_{v}(t) / \sum_{u=1}^{V} I_{u}(t)$, we can derive an ODE for variant frequency
\begin{align*}
    \frac{d f_{v}}{d t} &= f_{v} \left( \sum_{u=1}^{V} [r_{v}(t) - r_{u}(t)] f_{u} \right)\\
                        &= f_{v} \left( r_{v}(t) - \sum_{u=1}^{V} r_{u}(t) f_{u} \right).
\end{align*}

We can see these equations have the following solution
\begin{align}
    f_{v}(t) &= \frac{ f_{v}(0) \exp( \int_{0}^{t} r_{v}(s) ds)}{\sum_{u=1}^{V}  f_{u}(0) \exp( \int_{0}^{t} r_{u}(s) ds)}
\end{align}


\paragraph{Relative frequency and relative fitness}%

Using the above, we can write the relative frequency of variant $v$ over $u$ as $x_{v,u}(t) = f_{v}(t) / f_{u}(t)$
\begin{align*}
    x_{v, u}(t) = \frac{f_{v}(t)}{f_{u}(t)} &= \frac{f_{v}(0)}{f_{u}(0)} \exp \left( \int_{0}^{t} [r_{v}(s) - r_{u}(s)] ds \right)\\
                                            &=x_{v,u}(0)\exp \left( \int_{0}^{t} \lambda_{v,u}(s) ds \right).
\end{align*}

Notice this relative frequency change depends on the initial relative frequencies and the function $\lambda_{v,u}(t) = r_{v}(t) - r_{u}(t)$ which we'll call the relative fitness of $v$ over $u$.
We can notice that this relative frequency is exactly
\begin{align}
\lambda_{v, u}(t) =  \frac{d }{d t} \left[\log \left( x_{v,u}(t) \right) \right]
\end{align}

What else is there to discuss here. Cumulative relative frequency as an explanation for frequency change?

\paragraph{Cummulative relative-fitness and frequency change}

\subsection*{Applications to epidemic models}

\paragraph{Two-strain SIR}%

For simplicity, we will begin by analyzing a two-strain SIR based model in which the variant viruses can differ by increased intrinsic transmissibility (via $\eta_{T}$) and immune escape against wild-type immunity (via $\eta_{E}$).
This system of 4 ordinary differential equations is written in full below.

\begin{align*}
    \frac{d S}{d t} &= - \beta S I_{w} - \beta \eta_{T} S I_{v}\\ 
    \frac{d I_{w}}{dt} &= \beta S I_{v} - \gamma I_{w}\\
    \frac{d I_{v}}{dt} &= \beta \eta_{T} S I_{v} + \beta \eta_{T} \eta_{E} \phi_{w} I_{v} - \gamma I_{v}\\
    \frac{d \phi_{w}}{dt} &= - \beta \eta_{T} \eta_{E} \phi_{w} I_{v}
\end{align*}

We can consider epidemic models which decompose to the form presented in equation (ref).

%TODO: Analysis the tranmissibility-escape trade off by writing out the relative fitness directly.
%TODO: Show that reltive fitness of an emergening immune escape variant depends on the present immunity
%TODO: Additionally that a plane of fitness which depends on the at-risk populations

\paragraph{Models of immune escape against heterogeneous backgrounds}%

We'll now consider a model where all hosts are assumed to fall into one of $B$ immune background $\phi_{b}$ for $b =1, \ldots, B$.
We assume that infection by each variant $v$ then leaves recovered hosts into the corresponding immune background $b_{v}$.
Variant transmission then occurs via immune escape against a background leading to a matrix of escape rates $\vec{\eta} = \eta_{v,b}$ for variants $v$ and background $b$.

We can then write the system of ODEs as 
\begin{align*}
    \frac{d I_{v}}{dt} &= \beta \sum_{1\leq b \leq B} \eta_{v, b} \phi_{b} I_{v} - \gamma I_{v}, \quad v = 1, \ldots, V\\
    \frac{d \phi_{b}}{dt} &= - \beta \sum_{1\leq v \leq V} \eta_{v,b}\phi_{b} I_{v} +  \sum_{v:\ b_{v} = b} \gamma I_{v}
\end{align*}

For simplicity, we'll begin with a two variant system with vaccination, so there are three immune background $\phi_{\wt}, \phi_{\var},$ and $\phi_{\vac}$.

We'll consider a two-variant example with vaccination.

Is there a silly model here which will conform to all our expectations.

(Does this `increase' antigenic evolution? What does that even mean?)

\section*{Applications}

\paragraph{Interpreting common methods of relative fitness estimation}%

Multinomial logistic regression assumes. This provides estimates of the relative fitness compared to some background strain.
Converting this estimate to an estimate of transmission advantage (relative effective reproduction number) requires assuming a delta distribution of the generation time. (Show this at length)

This will discuss basic mlr and spline mlr... (What kind of ODEs do these beget?)
When is this model good? (When at-risk pop is barely changing?)

Comparing to our previous examples, the assumption of constant growth advantage is most appropriate 

\paragraph{Estimating relative fitness using Gaussian Processes}%

\paragraph{Transmisibility-Escape Tradeoff}%

\paragraph{Are initial growth rates sufficient for predicting dominance}%

What happens when there is no mutation? Practical limits given mutation dynamics?

\paragraph{How does changing immune pools affect selection?}%
- This is looking at vaccination as a driver of selection...

\paragraph{Correlations are insufficient for mechanism identification}%

\paragraph{Predictive capabilities of simple models on complex dynamics}%

\section*{Discussion}

\section*{Methods}

\subsection*{Data and code accessibility}

Source code used to generate figures are available at. 
Additionally, implementations of all models discussed are included at ...

\subsection*{Competing interests}%

\subsection*{Author contributions}
MF conceived the study. 
% TB gathered sequence and case count data. 
MF designed and implemented inference models. 
MF performed the analysis. 
% MF, TB interpreted the results. 
% MF, TB wrote the paper.


\subsection*{Acknowledgements}%

\cite{Ito2021}

\bibliographystyle{plos}
\bibliography{relative-fitness-mechanisms}

\newpage

\appendix

\setcounter{figure}{0}
\setcounter{table}{0}
\setcounter{page}{1}
\renewcommand{\thefigure}{S\arabic{figure}}
\renewcommand{\thetable}{S\arabic{table}}
\renewcommand{\thepage}{S\arabic{page}}

\section*{Supplemental Appendix}

\end{document}
