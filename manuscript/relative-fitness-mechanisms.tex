%Preamble
\documentclass[12pt,oneside,letterpaper]{article}

% graphicx package, useful for including eps and pdf graphics
\usepackage{graphicx}
\DeclareGraphicsExtensions{.pdf,.png,.jpg}

% basic packages
\usepackage{color}
\usepackage{parskip}
\usepackage{float}

% make links a more pleasing color and don't box them
\definecolor{blue}{rgb}{0.85,0.70,0.99}
\usepackage[hidelinks]{hyperref}
\hypersetup{colorlinks=true,linkcolor=black,citecolor=black,urlcolor=blue}

% text layout
\usepackage{geometry}
\geometry{textwidth=15cm} % 15.25cm for single-space, 16.25cm for double-space
\geometry{textheight=22cm} % 22cm for single-space, 22.5cm for double-space

% helps to keep figures from being orphaned on a page by themselves
\renewcommand{\topfraction}{0.85}
\renewcommand{\textfraction}{0.1}

% bold the 'Figure #' in the caption and separate it with a period
% Captions will be left justified
\usepackage[labelfont=bf,labelsep=period,font=small]{caption}

% review layout with double-spacing
%\usepackage{setspace}
%\doublespacing
%\captionsetup{labelfont=bf,labelsep=period,font=doublespacing}

% cite package, to clean up citations in the main text
\usepackage{cite}

% Remove brackets from numbering in list of References
\renewcommand\refname{\large References}
\makeatletter
\renewcommand{\@biblabel}[1]{\quad#1.}
\makeatother

\usepackage{authblk}
\renewcommand\Authands{ \& }
\renewcommand\Authfont{\normalsize \bf}
\renewcommand\Affilfont{\small \normalfont}
\makeatletter
\renewcommand\AB@affilsepx{, \protect\Affilfont}
\makeatother

% notation
\usepackage{amsmath}
\usepackage{amssymb}

\renewcommand{\vec}[1]{\boldsymbol{#1}}
\newcommand{\Prob}{\mathbb{P}}
\newcommand{\wt}{\text{wt}}
\newcommand{\var}{\text{var}}
\newcommand{\vac}{\text{vac}}


\title{Transmission mechanisms determine relative fitness and frequency dynamics of viral variants}

\author[1,2,*]{Marlin D.\ Figgins}
\author[1,3]{Trevor Bedford}
\affil[1]{Vaccine and Infectious Disease Division, Fred Hutchinson Cancer Research Center, Seattle, WA, USA}
\affil[2]{Department of Applied Mathematics, University of Washington, Seattle, WA, USA}
\affil[3]{Howard Hughes Medical Institute, Seattle, WA, USA}
\affil[*]{Corresponding author: mfiggins@uw.edu}
\date{\today}
%Figure out author affilations...
% https://maehler.se/blog/2013/11/01/authors-and-affiliations-in-latex

\begin{document}

\maketitle

\begin{abstract}
    Over the course of the COVID-19 pandemic, several SARS-CoV-2 variants have emerged globally, leading to large variant waves in populations of mixed immune backgrounds.
    Classifying these variant viruses into coarse fitness groups has allowed scientists to track the rise and fall of variants by modeling their frequencies over time.
    These models of frequency dynamics enable us to infer fitness advantages in variants, however, these models are typically divorced from a direct interpretation at the level of transmission.
    In this paper, we derive existing frequency dynamic models from exponentially growing populations and extend them to show that relative fitness of variants can be interpreted using compartment models of infectious diseases.
    We use these models to highlight several complications of when analyzing the frequency dynamics for infectious disease variants, namely the difficulty in discovering fitness mechanisms from frequency data alone. 
    We then develop several models for inference of frequency dynamics using Gaussian Processes and dimensionality reduction techniques to estimate latent pseudo-immune components which explain variant dynamics.
    Finally, we apply these models to SARS-CoV-2 sequences in the United States.
     % To better understand the relationship between population dynamics expected from traditional compartment models of epidemics,
\end{abstract}

\section*{Introduction}

The COVID-19 was characterized through large sweeping waves of new variants. 
This phenomenon is consistent with antigenic evolution and is observation in several other viruses such as [KATIE???]


To estimate and understand variant turn-over, frequency dynamic models of variant relative fitness have been developed.
These models estimate the relative fitness of circulating variants from frequency data, typically represented by counts of variant sequences over time.
Relative fitness in these models is often assumed to be a constant quantity for simplicity and intrinsic to the variant of interest, however this may not be reasonable due to the nature of the transmission process.

Additionally, translating these relative fitness estimates to a population-level transmission advantage (in terms of variant infections per wildtype infection) requires assumptions on the generation time of the transmission. [CITE]

%TODO: Mention that we need to make assumptions about transmission process (generation time) to convert to population level

It has been shown that variant transmission advantages differ geographically and temporally which suggests that variant transmission advantages are not necessarily fixed or determined intrinstically.
Existing models which do allow for variation in variant transmission advantages often do not have a mechanistic underpinning for why these transmission advantages exist and may persist.
These non-mechanistic transmission advantages are useful for real-time situation analysis though more work must be done to develop theory for how various mechanisms contribute to population change.

We introduce a theory of relative fitness and transmission advantages for exponentially growing populations.
We then apply this to several ordinary differential equation models of epidemics to assess how different transmission mechanisms may contribute to variant turnover and describe relative fitness.
Using these models, we then highlight the importance of population immunity, immune escape, and intrinsic transmissibility in determining variant turnover as well as short term forecasts of frequency growth.

With the knowledge gained from this analysis, we develop several new models of variant frequency dynamics which we use to...


\section*{Results}

\subsection*{Exponentially growing populations to frequency dynamics}%

We consider a viral population consisting of $V$ exponentially-growing variants with prevalence $I_{v}$ which follows:
\begin{align*}
    \frac{d I_{v}}{d t} = r_{v}(t) I_{v}(t), \quad v = 1,2, \ldots, V.
\end{align*}

Here, $r_{v}(t)$ is the in-homogenous exponential growth rate for variant $v$.

This differential equation has known solution

\begin{align*}
I_{v}(t) = I_{v}(0) \exp\left( \int_{0}^{t} r_{v}(s) ds\right),
\end{align*}
where $I_{v}(0)$ is the initial number of infectious individuals of variant $v$. 

Writing the frequency of variant $v$ in the population as  $f_{v}(t) = I_{v}(t) / \sum_{u=1}^{V} I_{u}(t)$, we can derive an ODE for variant frequency
\begin{align*}
    \frac{d f_{v}}{d t} &= f_{v} \left( \sum_{u=1}^{V} [r_{v}(t) - r_{u}(t)] f_{u} \right)\\
                        &= f_{v} \left( r_{v}(t) - \sum_{u=1}^{V} r_{u}(t) f_{u} \right).
\end{align*}

We can see these equations have the following solution
\begin{align}
    f_{v}(t) &= \frac{ f_{v}(0) \exp( \int_{0}^{t} r_{v}(s) ds)}{\sum_{u=1}^{V}  f_{u}(0) \exp( \int_{0}^{t} r_{u}(s) ds)}
\end{align}


\paragraph{Relative frequency and relative fitness}%

Using the above, we can write the relative frequency of variant $v$ over $u$ as $x_{v,u}(t) = f_{v}(t) / f_{u}(t)$
\begin{align*}
    x_{v, u}(t) = \frac{f_{v}(t)}{f_{u}(t)} &= \frac{f_{v}(0)}{f_{u}(0)} \exp \left( \int_{0}^{t} [r_{v}(s) - r_{u}(s)] ds \right)\\
                                            &=x_{v,u}(0)\exp \left( \int_{0}^{t} \lambda_{v,u}(s) ds \right).
\end{align*}

Notice this relative frequency change depends on the initial relative frequencies and the relative fitness $\lambda_{v,u}(t) = r_{v}(t) - r_{u}(t)$ of $v$ over $u$.
We can notice that this relative frequency is exactly
\begin{align}
\lambda_{v, u}(t) = r_{v}(t) - r_{u}(t) = \frac{d }{d t} \left[\log \left( x_{v,u}(t) \right) \right] = - \lambda_{u,v}(t)
\end{align}

This definition of relative fitness becomes essential in describing various modeling approaches for frequency dynamic data.

\paragraph{Cumulative relative-fitness and frequency change}

From this, we can see that frequency change over time intervals depends only on the cumulative relative fitness over time intervals.
Models of frequency change then can be described in terms of how they represent and estimate these relative fitnesses.
This framework includes various existing methods for analyzing frequency data such as Huddleston et al, MLR, MLR Spline, ...

In the following section, we show this framework applied to several simple compartmental models of epidemics.

\subsection*{Applications to epidemic models}

\paragraph{Two-strain SIR}%

For simplicity, we will begin by analyzing a two-strain SIR based model in which the variant viruses can differ by increased intrinsic transmissibility (via $\eta_{T}$) and immune escape against wild-type immunity (via $\eta_{E}$).
This system of 4 ordinary differential equations is written in full below.

\begin{align*}
    \frac{d S}{d t} &= - \beta S I_{w} - \beta \eta_{T} S I_{v}\\ 
    \frac{d I_{w}}{dt} &= \beta S I_{v} - \gamma I_{w}\\
    \frac{d I_{v}}{dt} &= \beta \eta_{T} S I_{v} + \beta \eta_{T} \eta_{E} \phi_{w} I_{v} - \gamma I_{v}\\
    \frac{d \phi_{w}}{dt} &= - \beta \eta_{T} \eta_{E} \phi_{w} I_{v}
\end{align*}

We can consider epidemic models which decompose to the form presented in equation (ref).

%TODO: Analysis the tranmissibility-escape trade off by writing out the relative fitness directly.
%TODO: Show that reltive fitness of an emergening immune escape variant depends on the present immunity
%TODO: Additionally that a plane of fitness which depends on the at-risk populations

\paragraph{Models of immune escape against heterogeneous backgrounds}%

We'll now consider a model where all hosts are assumed to fall into one of $B$ immune background $\phi_{b}$ for $b =1, \ldots, B$.
We assume that infection by each variant $v$ then leaves recovered hosts into the corresponding immune background $b_{v}$.
Variant transmission then occurs via immune escape against a background leading to a matrix of escape rates $\vec{\eta} = \eta_{v,b}$ for variants $v$ and background $b$.

We can then write the system of ODEs as 
\begin{align*}
    \frac{d I_{v}}{dt} &= \beta \sum_{1\leq b \leq B} \eta_{v, b} \phi_{b} I_{v} - \gamma I_{v}, \quad v = 1, \ldots, V\\
    \frac{d \phi_{b}}{dt} &= - \beta \sum_{1\leq v \leq V} \eta_{v,b}\phi_{b} I_{v} +  \sum_{v:\ b_{v} = b} \gamma I_{v}
\end{align*}

With this formulation, we can then write the relative fitnesses in terms of the escape rates $\vec{\eta}$ and the immune background proportions $\phi_{b}$ :

\begin{align*}
    \lambda_{v, u}(t) = \sum_{1\leq v \leq B}(\eta_{v,b} - \eta_{u,b}) \phi_{b}(t).
\end{align*}

This suggests that the relative fitnesses among variants can be decomposed into differences in immune escape among immune backgrounds within a population.
Understanding the size and complexity of this immune space may therefore be useful for parameterization and forecasting on variant frequencies.

%TODO: Suggest that we might use PCA, SVD, and latent factor models to gain a better understanding of these models

For simplicity, we'll begin with a two variant system with vaccination, so there are three immune background $\phi_{\wt}, \phi_{\var},$ and $\phi_{\vac}$.

(Does this `increase' antigenic evolution? What does that even mean?)

\section*{Applications}

Using the theory developed for exponentially-growing variant populations, we now re-visit existing methods for modeling viral frequency dynamics.

\paragraph{Multinomial Logistic Regression}%

We begin with multinomial logistic regression with fixed relative fitnesses (MLR).
This model can be written as

\begin{align*}
    f_{v}(t) = \frac{f_{v}(0) \exp(\lambda_{v} t)}{\sum_{u} f_{u}(0) \exp(\lambda_{u} t)},
\end{align*}

where $f_{v}(t)$ is the frequency of variant $v$ at time $t$ and $\lambda_{v}$ is the relative fitness of variant $v$.
This provides estimates of the relative fitness compared to some reference strain $u^{*}$for which $\lambda_{u^*} = 0$.
Converting this estimate to an estimate of transmission advantage (relative effective reproduction number) requires assuming a delta distribution of the generation time (CITE FIGGINS AND BEDFORD)

We can then see this model results from assuming that the at-risk populations are constant over-time.
This assumption is useful since it requires no outside knowledge of the at-risk population and relative infection rates, though this may be less useful for longer forecasts or when there is large turnover in at-risk populations due to infection.

\paragraph{Flu-forecasting}%

Motivated by the observed antigenic evolution of seasonal influenza, the authors approximate the cumulative relative fitness between flu seasons on the level of individual strains as:

\begin{align*}
    \Lambda_{v,u}(t + \Delta t,t) = \beta_{1} x_{v,1} + \cdots + \beta_{p} x_{v, p} (\Delta t) = \vec{\beta} \cdot \vec{x}_{v} (\Delta t),
\end{align*}

where the relative fitness is determined by strain-specific predictors $\vec{x}_{v}$ and the regression parameter $\vec{\beta}_{v}$ are estimated.
The authors choose predictors which describe the antigenic properties of strains though our previous work in 

\paragraph{Estimating relative fitness using Gaussian Processes}%

We develop a method for using Gaussian processes to model variant relative fitnesses.
Here, the relative fitnesses are modeled using a Gaussian process with a ... kernel and shared hyper-parameters.
This model is used in Figure ... to estimate the ... in ...

\paragraph{Transmisibility-Escape Tradeoff}%

\paragraph{Are initial growth rates sufficient for predicting dominance}%

What happens when there is no mutation? Practical limits given mutation dynamics?
Robustness of relative fitness under similar and varying mechanisms.

\paragraph{How does changing immune pools affect selection?}%
- This is looking at vaccination as a driver of selection...

\paragraph{Correlations are insufficient for mechanism identification}%

\paragraph{Predictive capabilities of simple models on complex dynamics}%

\paragraph{Dimensionality of immune space}%

\paragraph{Latent factor models of relative fitness}

\section*{Discussion}

\section*{Methods}

\subsection*{Data and code accessibility}

Source code used to generate figures are available at. 
Additionally, implementations of all models discussed are included at ...

\subsection*{Competing interests}%

\subsection*{Author contributions}
MF conceived the study. 
% TB gathered sequence and case count data. 
MF designed and implemented inference models. 
MF performed the analysis. 
% MF, TB interpreted the results. 
% MF, TB wrote the paper.


\subsection*{Acknowledgements}%

\cite{Ito2021}

\bibliographystyle{plos}
\bibliography{relative-fitness-mechanisms}

\newpage

\appendix

\setcounter{figure}{0}
\setcounter{table}{0}
\setcounter{page}{1}
\renewcommand{\thefigure}{S\arabic{figure}}
\renewcommand{\thetable}{S\arabic{table}}
\renewcommand{\thepage}{S\arabic{page}}

\section*{Supplemental Appendix}

\end{document}
