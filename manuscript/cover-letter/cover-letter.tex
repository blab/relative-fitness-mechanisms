\documentclass[11pt]{article}

\usepackage{geometry}
\geometry{left=1.1in, right=1.1in, top=1.0in, bottom=0.95in}

% graphicx package, useful for including eps and pdf graphics
\usepackage{graphicx}
\DeclareGraphicsExtensions{.pdf,.png,.jpg}

% basic packages
\usepackage{color}
\usepackage{parskip}
\setlength{\parskip}{0.16cm}
\usepackage{float}
\usepackage{todonotes}
\usepackage{enumitem}
\usepackage{microtype}

\definecolor{green}{rgb}{0.20,0.50,0.48}
\usepackage[hidelinks]{hyperref}
\hypersetup{colorlinks=true,linkcolor=black,citecolor=black,urlcolor=green}

\setlength{\parindent}{0pt} % Remove paragraph indent

\definecolor{brown}{rgb}{0.700,0.150,0.150}
\def\mfc#1{\textcolor{brown}{[#1]}}

\begin{document}

\thispagestyle{empty} % Remove page headers/footers

\mbox{}\hfill
\begin{tabular}{l @{}}
	% \includegraphics[width=6.5cm]{figures/fhcrc_logo} \\
	Vaccine and Infectious Disease Division \\
	Fred Hutchinson Cancer Center \\
	1100 Fairview Ave N \\
	Seattle, WA 98109, USA \\
	Phone: (206) 667-6372 \\
	Email: \href{mailto:mfiggins@uw.edu}{mfiggins@uw.edu} \\
\end{tabular}

\vspace{0.1in} % Vertical skip between sender/receiver address

\begin{tabular}{@{} l}
  \today
\end{tabular}

\vspace{0.1in} % Vertical skip between receive address and letter opening

Dear Editors,

\medskip % Vertical skip between letter opening and letter body

Please find attached our manuscript entitled ``Transmission mechanisms and selective pressure shape viral fitness and growth''.
We would be grateful if you would consider it for publication in \textit{Science}.

Viral variants, like those seen during the COVID-19 pandemic, have captured the world’s attention.
However, understanding which variants will dominate and why has remained a challenge for scientists and public health officials alike.
Despite the repeated emergence of SARS-CoV-2 variants causing large waves of infection, surprisingly little work has focused on what shapes variant advantage and how these dynamics impact transmission.
Our research directly addresses this gap.

By integrating genetic data with key epidemiological forces-- transmissibility and immune escape-- we provide a model that explains the mechanisms driving variant success.
This allows us to understand how past exposure creates a trade-off between transmissibility increase and immune evasion for pathogens.
We introduce a novel method for estimating changes in variant fitness using Gaussian processes, which offers a flexible way to track time-varying fitness regardless of underlying mechanism.
Combined with our selective pressure metric, we can quantify the impact of genetic changes on population-level epidemic growth.
This enables us to predict transmission rates even in the absence of case data-- an increasingly important feature as case surveillance declines in much of the world.

Additionally, we develop a method motivated by models of immune escape that allows us to estimate approximate antigenic similarities between variants, providing insight into immune escape without the need to collect serological data.
This approach also allows us to estimate differences in immunity and escape risk of variants globally without the use of serological data.

As data availability declines, the ability to predict transmission rates and immunological similarity using sequences alone represents a major shift in how we can monitor evolving pathogens including SARS-CoV-2, influenza, and future viral threats.
We believe this manuscript will resonate with a broad and diverse audience, from researchers in evolutionary biology and epidemiology to public health professionals, policymakers, and anyone intrigued by how and why SARS-CoV-2 variant emerge and spread.
For this reason, we would greatly appreciate your consideration for publication in Science.

\vspace{0.3in} % Vertical skip between letter closing and signature start

Sincerely, \newline
\vspace{0.05in} \newline
Marlin Figgins \newline
PhD Candidate \newline
Department of Applied Mathematics, University of Washington \newline
Vaccine and Infectious Disease Division, Fred Hutchinson Cancer Center

\end{document}
